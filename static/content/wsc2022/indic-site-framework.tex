% Template - https://sanskrit.uohyd.ac.in/18WSC/Style_files/CS_and_DH.tex
\providecommand{\tightlist}{%
  \setlength{\itemsep}{0pt}\setlength{\parskip}{0pt}}

\documentclass[11pt]{article}
\usepackage{scl}
\usepackage{times}
\usepackage{url}
\usepackage{latexsym}
\usepackage{lineno}

\usepackage{fontspec, xunicode, xltxtra}
\newfontfamily\skt[Script=Devanagari]{Sanskrit 2003}
\setmonofont{Sanskrit 2003}


\title{A client-side-dynamic website framework for annotated Indic texts}

\author{
  Vishvas Vasuki \\
  Dyugaṅgā, Beṅgaḷūru \\
  {\tt https://sanskrit.github.io/groups/dyuganga/}
\\}

\date{}

\begin{document}
\maketitle
%\linenumbers
\begin{abstract}
Perusing Indic classics on the web is an increasingly popular activity. However, serving and maintaining feature-rich websites involves significant costs, knowledge and effort. We present an easy to use website framework to solve mitigate these drawbacks. Besides being simple and low-cost, the proposed solution is rich in features - including easy transliteration, navigation aids, annotation support, dynamic content inclusion and more.
\end{abstract}

\section{Motivation}
The advent of internet, personal computers and mobile devices have radically changed the way we consume literature. We can now carry an entire library in our pockets. Besides access to simple text, digital readers are able to utilize features such as text-to-speech, embedded multimedia, easy navigation and dictionary lookup. 

On the production end - publishing on the web is far simpler and cheaper than publishing paper books. Yet, doing it well (i.e. with the dynamic features digital readers expect) still involves significant costs, knowledge and effort. Furthermore, for highly structured texts - like originals with series of commentaries and translations - content management becomes non-trivial.

We present a framework to publish annotated Indic language texts, providing the following features:



\section{Static site generation framework}
Hugo is a popular static site generation framework \cite{hugo}.

\section{Client side dynamism}


% include your own bib file like this:
\bibliographystyle{acl}
\bibliography{indic-site-framework}
\end{document}